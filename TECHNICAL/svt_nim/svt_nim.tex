%% 
%% Copyright 2007, 2008, 2009 Elsevier Ltd
%% 
%% This file is part of the 'Elsarticle Bundle'.
%% ---------------------------------------------
%% 
%% It may be distributed under the conditions of the LaTeX Project Public
%% License, either version 1.2 of this license or (at your option) any
%% later version.  The latest version of this license is in
%%    http://www.latex-project.org/lppl.txt
%% and version 1.2 or later is part of all distributions of LaTeX
%% version 1999/12/01 or later.
%% 
%% The list of all files belonging to the 'Elsarticle Bundle' is
%% given in the file `manifest.txt'.
%% 

%% Template article for Elsevier's document class `elsarticle'
%% with numbered style bibliographic references
%% SP 2008/03/01

\documentclass[preprint,12pt]{elsarticle}

%% Use the option review to obtain double line spacing
%% \documentclass[authoryear,preprint,review,12pt]{elsarticle}

%% Use the options 1p,twocolumn; 3p; 3p,twocolumn; 5p; or 5p,twocolumn
%% for a journal layout:
%% \documentclass[final,1p,times]{elsarticle}
%% \documentclass[final,1p,times,twocolumn]{elsarticle}
%% \documentclass[final,3p,times]{elsarticle}
%% \documentclass[final,3p,times,twocolumn]{elsarticle}
%% \documentclass[final,5p,times]{elsarticle}
%% \documentclass[final,5p,times,twocolumn]{elsarticle}

%% For including figures, graphicx.sty has been loaded in
%% elsarticle.cls. If you prefer to use the old commands
%% please give \usepackage{epsfig}

%% The amssymb package provides various useful mathematical symbols
\usepackage{amssymb}
%% The amsthm package provides extended theorem environments
%% \usepackage{amsthm}

%% The lineno packages adds line numbers. Start line numbering with
%% \begin{linenumbers}, end it with \end{linenumbers}. Or switch it on
%% for the whole article with \linenumbers.
%% \usepackage{lineno}

\journal{Nuclear Physics B}

\begin{document}

\begin{frontmatter}

%% Title, authors and addresses

%% use the tnoteref command within \title for footnotes;
%% use the tnotetext command for theassociated footnote;
%% use the fnref command within \author or \address for footnotes;
%% use the fntext command for theassociated footnote;
%% use the corref command within \author for corresponding author footnotes;
%% use the cortext command for theassociated footnote;
%% use the ead command for the email address,
%% and the form \ead[url] for the home page:
%% \title{Title\tnoteref{label1}}
%% \tnotetext[label1]{}
%% \author{Name\corref{cor1}\fnref{label2}}
%% \ead{email address}
%% \ead[url]{home page}
%% \fntext[label2]{}
%% \cortext[cor1]{}
%% \address{Address\fnref{label3}}
%% \fntext[label3]{}

\title{The Heavy Photon Search Silicon Vertex Tracker}

%% use optional labels to link authors explicitly to addresses:
%% \author[label1,label2]{}
%% \address[label1]{}
%% \address[label2]{}

%\author{}
%
%\address{}

\newcommand{\red[1]}{{\color{red}{\bf #1}}}
\newcommand{\JLAB}{Thomas Jefferson National Accelerator Facility, Newport News, Virginia 23606}
\newcommand{\CUA}{Catholic University of America, Washington, D.C. 20064}
\newcommand{\OU}{Ohio University,  Athens, Ohio 45701}
\newcommand{\YEREVAN}{Yerevan Physics Institute, 375036 Yerevan, Armenia}
\newcommand{\SCAROLINA}{University of South Carolina, Columbia, South Carolina 29208}
\newcommand{\NSU}{Norfolk State University, Norfolk, Virginia 23504}
\newcommand{\ODU}{Old Dominion University, Norfolk, Virginia 23529}
\newcommand{\genova}{Istituto Nazionale di Fisica Nucleare, Sezione di Genova e Dipartimento di Fisica dell\'Universita, 16146 Genova, Italy}
\newcommand{\SACLAY}{CEA, Centre de Saclay, Irfu/Service de Physique Nucl\'eaire 91191 Gif-sur-Yvette, France}
\newcommand{\ORSAY}{Institut de Physique Nucleaire d'Orsay, IN2P3, BP 1, 91406 Orsay, France}
\newcommand{\ECOLE}{CPhT, Ecole Polytechnique, F 91128 PALAISEAU CEDEX, France}
\newcommand{\UCSC}{University of California, Santa Cruz, CA 95064}
\newcommand{\SUNY}{Stony Brook University, Stony Brook, NY 11794-3800}
\newcommand{\FNAL}{Fermi National Accelerator Laboratory, Batavia, IL 60510-5011}
\newcommand{\UNH}{University of New Hampshire, Department of Physics, Durham, NH 03824}
\newcommand{\PERIMETER}{Perimeter Institute, Ontario, Canada N2L 2Y5}
\newcommand{\RPI}{Rensselaer Polytechnic Institute, Department of Physics, Troy, NY 12181}
\newcommand{\SLAC}{SLAC National Accelerator Laboratory, Menlo Park, CA 94025}
\newcommand{\WNM}{The College of William and Mary, Department of Physics, Williamsburg, VA 23185}


\author[SLAC]{C. Field}
\author[SLAC]{P. Hansson Adrian\corref{corrauthor}}
\author[SLAC]{N. Graf} 
\author[SLAC]{ M. Graham} 
\author[SLAC]{ G. Haller} 
\author[SLAC]{ R. Herbst} 
%\author[SLAC]{ J. Jaros\corref{spoks}} 
\author[SLAC]{ J. Jaros}
\author[SLAC]{T. Maruyama} 
\author[SLAC]{ J. McCormick} 
\author[SLAC]{ K. Moffeit} 
\author[SLAC]{ T. Nelson} 
\author[SLAC]{ H. Neal} 
\author[SLAC]{ A. Odian} 
\author[SLAC]{ M. Oriunno} 
\author[SLAC]{ S. Uemura} 
\author[SLAC]{ D. Walz}
%
\author[UCSC]{A. Grillo} 
\author[UCSC]{ V. Fadeyev} 
\author[UCSC]{ O. Moreno}
%
\author[FNAL]{W. Cooper}
%
\author[JLAB]{S. Boyarinov} 
\author[JLAB]{ V. Burkert} 
\author[JLAB]{ A. Deur} 
\author[JLAB]{ H. Egiyan}
 \author[JLAB]{ L. Elouadrhiri} 
 \author[JLAB]{ A. Freyberger} 
 \author[JLAB]{ F.-X. Girod} 
 \author[JLAB]{ V. Kubarovsky} 
\author[JLAB]{ Y. Sharabian} 
%\author[JLAB]{ S. Stepanyan\corref{spoks}} 
\author[JLAB]{ S. Stepanyan}
\author[JLAB]{ M. Ungaro} 
\author[JLAB]{ B. Wojtsekhowski}
%
\author[SUNY]{R. Essig}
%
%\author[UNH]{M. Holtrop\corref{spoks}} 
\author[UNH]{M. Holtrop}
\author[UNH]{ K. Slifer} 
\author[UNH]{ S. K. Phillips}
%
\author[ORSAY]{A. Fradi} 
\author[ORSAY]{ B. Guegan} 
\author[ORSAY]{ M. Guidal} 
\author[ORSAY]{ S. Niccolai} 
\author[ORSAY]{ S. Pisano} 
\author[ORSAY]{ E. Rauly}
 \author[ORSAY]{ P. Rosier and D. Sokhan}
 %
\author[PERIMETER]{P. Schuster} 
\author[PERIMETER]{ N. Toro}
%
\author[YEREVAN]{N. Dashyan} 
\author[YEREVAN]{ N. Gevorgyan} 
\author[YEREVAN]{ R. Paremuzyan}
\author[YEREVAN]{ H. Voskanyan}
%
\author[NSU]{M. Khandaker} 
\author[NSU]{ C. Salgado}
%
\author[GENOVA]{M. Battaglieri} 
\author[GENOVA]{ R. De Vita}
%
\author[ODU]{S. Bueltmann} 
\author[ODU]{ L. Weinstein}
%
\author[RPI]{P. Stoler} 
\author[RPI]{A. Kubarovsky}
%
\author[WNM]{K. Griffioen}

\address[SLAC]{\SLAC}                                 
\address[UCSC]{\UCSC}
\address[FNAL]{\FNAL}
\address[JLAB]{\JLAB}
\address[SUNY]{\SUNY}
\address[UNH]{\UNH}
\address[ORSAY]{\ORSAY}
\address[PERIMETER]{\PERIMETER}
\address[YEREVAN]{\YEREVAN}
\address[NSU]{\NSU}
\address[GENOVA]{\genova}
\address[ODU]{\ODU}
\address[RPI]{\RPI}
\address[WNM]{\WNM}

 %\cortext[spoks]{Co-spokesperson}

\cortext[corrauthor]{Corresponding author. \ead{phansson@slac.stanford.edu}}



\begin{abstract}
%% Text of abstract

\end{abstract}

\begin{keyword}
%% keywords here, in the form: keyword \sep keyword

%% PACS codes here, in the form: \PACS code \sep code

%% MSC codes here, in the form: \MSC code \sep code
%% or \MSC[2008] code \sep code (2000 is the default)

\end{keyword}

\end{frontmatter}

%% \linenumbers

%% main text

\section{Introduction}
\label{Introduction}

\section{SVT Design}
\label{Design}

\subsection{SVT Layout (Tim)}
\label{Layout}

\subsection{Sensors and Front End Readout (Tim)}
\label{SensorsFE}

\subsection{Module Design (Tim)}
\label{ModuleDesign}

\subsection{Mechanics and Cooling (Tim)}
\label{MechanicsCooling}

\subsection{Data Acquisition (Pelle and Ben)}
\label{DAQDesign}

\subsection{Monitoring and Controls (Tim and Sho)}
\label{MonitoringControls}

\section{SVT Assembly, Testing, and Installation}
\label{Assembly}

\subsection{Module Production and Testing (Tim and Omar)}
\label{ModuleProduction}

\subsection{Mechanical Assembly and Testing (Tim and Sho)}
\label{MechanicalAssembly}

\subsection{DAQ Assembly and Testing (Pelle and Ben)}
\label{DAQAssembly}

\subsection{Full System Testing (Tim)}
\label{SystemTesting}

\subsection{Shipping and Installation (Tim)}
\label{ShippingInstallation}

\section{SVT Operation and Performance}
\label{Performance}

\subsection{SVT Operation (Tim)}
\label{Operation}

\subsection{SVT Hit Performance (Omar and Matt S.)}
\label{HitPerformance}

\subsection{SVT Tracking Performance (Sho and Matt G.)}
\label{TrackingPerformance}

\subsection{SVT Vertex Performance (Sho and Omar)}
\label{VertexPerformance}

\section{Summary and Outlook (Tim)}
\label{Summary}

\section{Acknowledgements}
\label{Acknowledgements}

%% The Appendices part is started with the command \appendix;
%% appendix sections are then done as normal sections
%% \appendix

%% \section{}
%% \label{}

%% If you have bibdatabase file and want bibtex to generate the
%% bibitems, please use
%%
%%  \bibliographystyle{elsarticle-num} 
%%  \bibliography{<your bibdatabase>}

%% else use the following coding to input the bibitems directly in the
%% TeX file.

\begin{thebibliography}{00}

%% \bibitem{label}
%% Text of bibliographic item

\bibitem{}

\end{thebibliography}
\end{document}
\endinput
%%
%% End of file `elsarticle-template-num.tex'.
