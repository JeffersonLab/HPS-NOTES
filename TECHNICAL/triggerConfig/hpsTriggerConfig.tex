\documentclass[amsmath,amssymb,notitlepage,11pt]{revtex4-1}
\usepackage{graphicx}
\usepackage{bm}% bold math
\usepackage{multirow}
\usepackage{booktabs}
\usepackage{verbatim}
\usepackage{hyperref}
\usepackage{setspace}
\usepackage{enumitem}
\hypersetup{pdftex,colorlinks=true,allcolors=blue}
\usepackage{hypcap}
\renewcommand{\baselinestretch}{1.0}
%\usepackage[small,compact]{titlesec}
%\usepackage{showkeys}
%\addtolength{\textheight}{0.3cm}
%\addtolength{\topmargin}{-0.15cm}
%\addtolength{\textwidth}{0.4cm}
%\addtolength{\hoffset}{-0.2cm}
\begin{document}
%\hspace*{11.5cm}\texttt{HPS-NOTE 2015-XXX}

\title{HPS Trigger Configuration}
\author{N. Baltzell}
\affiliation{Jefferson Lab}
\date{\today}
\begin{abstract}
Here we document the format and location of the DAQ and trigger configuration files as used in HPS's 2015 Engineering Run. 
\end{abstract}
\maketitle

\section{Configuration File Overview}
The configuration files are plain ascii with a format specific to Hall-B DAQ.  They contain one key/value(s) pair per line.  Each crate reads the full configuration but only uses the lines delimited by tags applicable to its crate.  

The DAQ only reads files from the base path of $\$CLON\_PARMS/trigger$, so the DAQ GUI used by shifters starts from there. 

Some specifics on the configuration files:
\begin{itemize}
    \item The \# character at the beginning of a line deontes a comment;  any line starting with a \# is completely ignored.
    \item Energies are always in units MeV, with no exceptions.
    \item Care must be taken with time units;  it can be either ns or \# samples, depending on who wrote the firmware for the given hardware.
    \item While each crate reads the entire configuration, they only interpret lines delimited by tags for its crate.
    \item In the case of multiple lines with equal keys, precendence always goes to the the last entry in the configuration file.
\end{itemize}

\section{GTP Settings}
The GTP controls the clustering algorithm.  The only configurable GTP parameters are clustering seed energy threshold and hit timing coincidence requirements.  The energy threshold is in units MeV (in the example below it is 50 MeV).  The two values for coincidence are number of FADC samples before and after seed hit (in the example below it is $\pm 4$ samples = $\pm$ 16 ns), and  currently the firmware requires these be equal.
\begin{verbatim}
GTP_CRATE all
GTP_CLUSTER_PULSE_THRESHOLD 50
GTP_CLUSTER_PULSE_COIN 4 4
GTP_CRATE end
\end{verbatim}

\section{TI Settings}
All TI trigger settings are defined in the TI section of the DAQ configuration files for crate \texttt{hps11}, between the following two lines:
\begin{verbatim}
TI_CRATE hps11
TI_CRATE end
\end{verbatim}

\subsection{Buffer and Block Sizes}
\begin{verbatim}
TI_FIBER_DELAY_OFFSET 0x80 0xcf
\end{verbatim}

Events are rea
\begin{verbatim}
TI_BUFFER_LEVEL 5
TI_BLOCK_LEVEL 10
\end{verbatim}

\subsection{Holdoff}
The deadtime of the TI can be configured 

\subsection{Prescales}
The six triggers can be independently prescaled in powers of 2.  The TI numbers the trigger from 1 to 6:
\begin{table}[htbp]\centering
  \begin{tabular}{lr}\toprule[1.5pt]
    TI Input \#\ \ \ \ \ \ \ \ \  & Trigger Type \\ \cmidrule[0.5pt]{1-2}
    1 & {\em singles-0} \\
    2 & {\em singles-1} \\
    3 & {\em pairs-0} \\
    4 & {\em pairs-1} \\
    5 & {\em calibration} \\
    6 & {\em pulser} \\
  \bottomrule[1.5pt]
  \end{tabular}
  \caption{TI trigger input numbering.}
\end{table}

The following lines in the trigger configuration file would set the {\em singles-0} prescale to $2^{13}$, the {\em singles-1} and {\em pairs-0} triggers' prescales to $2^{11}$, and leave all other trigger with no prescaling: 
\begin{verbatim}
TI_INPUT_PRESCALE 1 13
TI_INPUT_PRESCALE 2 11
TI_INPUT_PRESCALE 3 11
TI_INPUT_PRESCALE 4  0
TI_INPUT_PRESCALE 5  0
TI_INPUT_PRESCALE 6  0
\end{verbatim}

\section{SSP Settings}
\begin{table}[htbp]\centering
  \begin{tabular}{llr}\toprule[1.5pt]
    SSP Input \#\ \ \ \ \ \ \ \ \  & SSP Output \# \ \ \ \ \ \ \ \ & Trigger Type \\ \cmidrule[0.5pt]{1-3}
    7 & 20 & {\em singles-0} \\
    8 & 21 & {\em singles-1} \\
    9 & 22 & {\em pairs-0} \\
    10 & 23 & {\em pairs-1} \\
    11 & 24/25& {\em calibration}\\
    12 & 18 & {\em pulser} \\
  \bottomrule[1.5pt]
  \end{tabular}
  \caption{SSP trigger numbering.  The two {\em calibration} IOs are LED/COSMIC, respectively. }
\end{table}

All SSP trigger settings are defined in the SSP section of the DAQ configuration files for crate \texttt{hps11}, between the following two lines:
\begin{verbatim}
SSP_CRATE hps11
SSP_CRATE end
\end{verbatim}

\subsection{FADC Trigger Window}
The trigger reads from the FADC pipeline, with a window and latency defined in terms of number of FADC samples (4 ns):
\begin{verbatim}
SSP_W_WIDTH   50
SSP_W_OFFSET  757
SSP_HPS_LATENCY 475
\end{verbatim}

\section{Enabling/Disabling Triggers}
All triggers are enabled and disabled with similar configuration lines, shown in :
\begin{verbatim}
SSP_HPS_SET_IO_SRC   7 20  # SINGLE-0 ENABLED
SSP_HPS_SET_IO_SRC   7  0  # SINGLE-0 DISABLED
SSP_HPS_SET_IO_SRC   8 21  # SINGLE-1 ENABLED
SSP_HPS_SET_IO_SRC   8  0  # SINGLE-1 DISABLED
SSP_HPS_SET_IO_SRC   9 22  # PAIR-0 ENABLED
SSP_HPS_SET_IO_SRC   9  0  # PAIR-0 DISABLED
SSP_HPS_SET_IO_SRC  10 23  # PAIR-1 ENABLED
SSP_HPS_SET_IO_SRC  10  0  # PAIR-1 DISABLED
SSP_HPS_SET_IO_SRC  11 24  # CALIB LED ENABLED
SSP_HPS_SET_IO_SRC  11 25  # CALIB COSMIC ENABLED
SSP_HPS_SET_IO_SRC  11  0  # CALIB DISABLED
SSP_HPS_SET_IO_SRC  12 18  # PULSER ENABLED
SSP_HPS_SET_IO_SRC  12  0  # PULSER DISABLED
\end{verbatim}

\section{Pulser Trigger}
There are three configuration lines that control the ``random'' pulser trigger.  The frequency of the pulser trigger is defined in Hz.  This line sets the pulser rate to 100 Hz:
\begin{verbatim}
SSP_HPS_PULSER 100
\end{verbatim}
This line will enable the pulser trigger:
\begin{verbatim}
SSP_HPS_SET_IO_SRC  12 18  # PULSER ENABLED
\end{verbatim}
and this line will disable the pulser trigger:
\begin{verbatim}
SSP_HPS_SET_IO_SRC  12  0  # PULSER DISABLED
\end{verbatim}

\section{Singles Triggers}
Here is how to enable and disable the two singles triggers:
\begin{verbatim}
SSP_HPS_SET_IO_SRC   7 20  # SINGLE-0 ENABLED
SSP_HPS_SET_IO_SRC   7  0  # SINGEL-0 DISABLED
SSP_HPS_SET_IO_SRC   8 21  # SINGLE-1 ENABLED
SSP_HPS_SET_IO_SRC   8  0  # SINGLE-1 DISABLED
\end{verbatim}
There are three possible cuts for each singles trigger:
\begin{enumerate}
  \item Minimum cluster energy.
  \item Maximum cluster energy.
  \item Minimum number of hits in a cluster.
\end{enumerate}
The first column denotes which singles trigger, 0 or 1.
The second column is the cut value.
The last column is 0/1 for disabled/enabled.
\begin{verbatim}
# Singles 0 trigger
SSP_HPS_SINGLES_EMIN  0  60   1
SSP_HPS_SINGLES_EMAX  0  2500 1
SSP_HPS_SINGLES_NMIN  0  3    1

# Singles 1 trigger
SSP_HPS_SINGLES_EMIN  1  400  1
SSP_HPS_SINGLES_EMAX  1  1100 1
SSP_HPS_SINGLES_NMIN  1  3    1
\end{verbatim}

\section{Pairs Triggers}
Here is how to enable and disable the two pairs triggers:
\begin{verbatim}
SSP_HPS_SET_IO_SRC    9 22  # PAIR-0 ENABLED
SSP_HPS_SET_IO_SRC    9  0  # PAIR-0 DISABLED
SSP_HPS_SET_IO_SRC   10 23  # PAIR-1 ENABLED
SSP_HPS_SET_IO_SRC   10  0  # PAIR-1 DISABLED
\end{verbatim}

There are nine possible cuts for each pair trigger:
\begin{enumerate}[itemsep=0mm]
  \item Minimum cluster energy.
  \item Maximum cluster energy.
  \item Minimum number of hits in a cluster.
  \item Minimum energy sum.
  \item Maximum energy sum.
  \item Maximum energy difference.
  \item Maximum coplanarity angle.
  \item Minimum energy/distance product.
\end{enumerate}
The first column denotes which singles trigger, 0 or 1.
The second column is the cut value.
The last column is 0/1 for disabled/enabled.
\begin{verbatim}
# Pairs 0 trigger
SSP_HPS_PAIRS_TIMECOINCIDENCE  0  4
SSP_HPS_PAIRS_EMIN             0  54
SSP_HPS_PAIRS_EMAX             0  1100
SSP_HPS_PAIRS_NMIN             0  1
SSP_HPS_PAIRS_SUMMAX_MIN       0  2000 120 1
SSP_HPS_PAIRS_DIFFMAX          0  1000     1
SSP_HPS_PAIRS_COPLANARITY      0  180      0
SSP_HPS_PAIRS_ENERGYDIST       0  5.5  100 0

# Pairs 1 trigger
SSP_HPS_PAIRS_TIMECOINCIDENCE  1  3
SSP_HPS_PAIRS_EMIN             1  54
SSP_HPS_PAIRS_EMAX             1  630
SSP_HPS_PAIRS_NMIN             1  1
SSP_HPS_PAIRS_SUMMAX_MIN       1  860 180  1
SSP_HPS_PAIRS_DIFFMAX          1  540      1
SSP_HPS_PAIRS_COPLANARITY      1  30       1
SSP_HPS_PAIRS_ENERGYDIST       1  5.5  600 1
\end{verbatim}

\section{Calibration Trigger}
This trigger can be used to trigger on the LED pulser or a coincidence of scintillators for cosmic calibration.
\begin{verbatim}
# HPS CALIBRATION COSMIC/LED -> TI TS5
#SSP_HPS_SET_IO_SRC  11 24  # CALIB LED ENABLED
#SSP_HPS_SET_IO_SRC  11 25  # CALIB COSMIC ENABLED
SSP_HPS_SET_IO_SRC   11  0  # CALIB DISABLED

# coinc time 10=40 ns
SSP_HPS_COSMIC_TIMECOINCIDENCE  10

# cosmic B0 and B1 (136<<8) + led trigger (254<<0)
SSP_HPS_COSMIC_PATTERNCOINCIDENCE   35070
\end{verbatim}


{\setstretch{1.0}
\begin{table}[htbp]\centering
  \begin{tabular}{llr}\toprule[1.5pt]
Clustering\ \ \ \ \ \ \ \ \ \ 
&Seed Energy Threshold&        50 MeV\\
&Hit Threshold&   12 ADC\\
&Hit Time Coincidence&         +/-16 ns\\
&FADC Window Size&     200 ns\\ \cmidrule[0.5pt]{1-3}
Singles-0
&Hits Per Cluster Min&             3\\
&Cluster Energy Min&              60 MeV\\
&Cluster Energy Max&            2500 MeV\\
&Prescale& $2^{13}$ \\ \cmidrule[0.5pt]{1-3}
Singles-1
&Hits Per Cluster Min&             3\\
&Cluster Energy Min&             400 MeV\\
&Cluster Energy Max&            1100 MeV\\
&Prescale& $2^{11}$ \\ \cmidrule[0.5pt]{1-3}
Pairs-0
&Hits Per Cluster Min&             1\\
&Cluster Time Coincidence&     +/-16 ns\\
&Cluster Energy Min&              54 MeV\\
&Cluster Energy Max&            1100 MeV\\
&2-Cluster Energy Sum Min&       120 MeV\\
&2-Cluster Energy Sum Max&      2000 MeV\\
&2-Cluster Energy Diff Max&     1000 MeV\\
&Coplanarity Max&                n/a\\
&Energy-Dist Slope&              n/a\\
&Energy-Dist Min&                n/a\\
&Prescale& $2^{11}$ \\ \cmidrule[0.5pt]{1-3}
Pairs-1
&Hits Per Cluster Min&             1\\
&Cluster Time Coincidence&     +/-12 ns\\
&Cluster Energy Min&              54 MeV\\
&Cluster Energy Max&             630 MeV\\
&2-Cluster Energy-Sum Min&       180 MeV\\
&2-Cluster Energy-Sum Max&       860 MeV\\
&2-Cluster Energy-Diff Max&      540 MeV\\
&Coplanarity Maximum&             30 deg\\
&Energy-Dist Slope&              5.5 MeV/mm\\
&Energy-Dist Minimum&            600 MeV\\
&Prescale& $2^{0}$ \\ \cmidrule[0.5pt]{1-3}
Pulser
&Rate&        100 Hz\\
&Prescale& $2^{0}$ \\ \cmidrule[0.5pt]{1-3}
  \bottomrule[1.5pt]
  \end{tabular}
  \caption{Configuration for the V7 trigger used in HPS's 2015 1.056 GeV Commissioning Run.}
\end{table}
}

\end{document}

