\documentclass[11pt]{article}
\usepackage{geometry}                % See geometry.pdf to learn the layout options. There are lots.
\geometry{letterpaper}                   % ... or a4paper or a5paper or ... 
%\geometry{landscape}                % Activate for for rotated page geometry
%\usepackage[parfill]{parskip}    % Activate to begin paragraphs with an empty line rather than an indent
\usepackage{graphicx}
\usepackage{amssymb}
\usepackage{epstopdf}
\usepackage{longtable}
\usepackage{amsmath}
\usepackage{amssymb}

\usepackage{placeins}
\usepackage{harpoon}

\usepackage{longtable}

\usepackage[square,numbers]{natbib}

\DeclareGraphicsRule{.tif}{png}{.png}{`convert #1 `dirname #1`/`basename #1 .tif`.png}

%\input{Macros.tex}
\title{HPS Note:  Golden Run Selection in 2016 Data}
\author{Sebouh Paul}
\date{}                                           % Activate to display a given date or no date

\begin{document}
\maketitle

\section{Abstract}
Here, I present a list of all of the ``golden runs" to be used in analyses on the HPS 2016 dataset.  For most of these runs, the entire run can be included when analyzing data.   However, there are a few runs on this list in which there were unusual conditions during part of the run which either make it difficult to get an accurate measurement of the luminosity, or in which the data was taken with unacceptable conditions for analysis.  In those cases, I indicate which range of files should be excluded, while the remainder of the run is included in the golden list.  


\section{Preliminaries}
The first step I took towards determining the golden runs was to look through the run spreadsheet to find a list of all of the production runs with version 7 or 8 trigger, either 150 or 200 nA, and the 4 mm tungsten target in place.  I excluded the runs that the shift takers labeled as ``junk" in the spreadsheet (except for run \# 7807, where it appeared that the shift-taker meant to apply the word ``junk" to the previous run but clicked on the wrong cell.).  



%\section{Ecal Peaks}
%From these runs, I looked at the shapes of the FEE peaks

\section{Runs and Parts of Runs Excluded}
\label{sec:exclude}
There were a few runs in which the SVT was not at the 0.5 mm position for part of the run.  This generally occurred when we started up the beam for the first time each weekend, and we first checked to see if the beam was stable before moving the SVT to its nominal position.  Among the runs where this happened, only run 7779 had a large enough amount of data at 0.5 mm that I decided to keep the part of this run with the SVT at 0.5 mm.  

 
%I may eventually add them to the golden-run list in the future, excluding the parts of the runs where the SVT is at the 1.5 mm position.  

There were also a few runs in which MYA recorded unusual values for the DAQ livetime (such as -999).  This may be a symptom of a worse problem in the DAQ for these runs, so therefore these are excluded in the golden list.  In runs 7961 and 8038, the mya values are nonsensical for all or a large portion of the run.  Runs 7962 and 8039 also have this problem, but only for the first few minutes, so I included these runs, but removed with the portions where the livetime from MYA was unacceptable.  Run 8043 had this problem for the first 2/3 of the run, but since the remaining third of the run had acceptable livetime, I decided to keep that part of the run.  

The beam blocker just upstream of the Faraday cup was removed during 2 minutes of run 7795.  This made the measurement of the beam charge during that run to be much higher than it actually was (the attenuation factor of the beam blocker is $\approx$ 50).   It is unlikely that this would have any effect on the data taken during that time.  However, I decided to remove that range of files without the beam blocker, to make the normalization easier.  

Run 7989 had no hits in the layer 3 top axial part of the SVT, according to the DQM.  In the recon, there are virtually no tracks found in the top half of the SVT.  The cause of this is unknown, but this may have been caused by a DAQ problem.  

Run 7973 has only one file and the bias was off for most of the run, therefore I excluded it.  

There were a few runs (7796, 7801, 7803, 7805, 7807) taken on March 6-7 in which a small number of files were missing.  I don't know if the reason for this is DAQ-related, or if there was a problem with copying them to tape.  I did not exclude these runs, since I did not find any other problems with them.  However, I do make note of the number of files missing in Table \ref{tbl:excluded_file_ranges} and account for the missing luminosity in Table \ref{tbl:runs_used}.  

\section{Normalization}
All of the luminosities are calculated using good gated beam charge with SVT bias.  For most of the runs, I simply used the beam charge from the run-summary sheet.  For the runs in which a few files were missing or in which I chose to exclude a range of files, I use the sum of the beam charges from the file-summary sheet of all of the files that I have included.  

%The luminosity of a run is given by $\ell = \frac{Q()}{(6.022\times10^{23})(10^{24}\textrm{ barn/cm}^2)}

The luminosity is given by $\ell = Q \left[\frac{\sigma N_A}{q_eA(10^{24}\textrm{ b/cm}^2)}\right]$, where $Q$ is the beam charge; $A$ is the atomic mass number (183.84 for tungsten); $N_A$ is Avagadro's number, $6.022\times 10^{23}\textrm{ mol}^2$; $\sigma$ is the target's areal density, 0.0078125 g/cm$^2$; and $q_e$ is the electron charge, $1.60217662 \times 10^{-19}$ C.  The quantity in brackets is thus 1.597e-4 nb$^{-1}$/nC.

\section{List of Golden Runs}
Table \ref{tbl:runs_used} lists the runs that I have selected.  Run numbers with asterisks next to them indicate that a certain range of the run is excluded from the golden run list and does not contribute to the luminosity or total events in the table.  These ranges of missing or excluded files is given in Table \ref{tbl:excluded_file_ranges}

\begin{longtable}[!hbtp]{l r r}
%\begin{center}

%  \tcaption{tbl:runs_used_bh}
%{List of ``golden" runs from the 2016 HPS physics run used in this analysis.}
%{The luminosities shown
%are corrected for livetime, periods where the SVT bias is off, and excludes data files with errors in the data acquisition and periods in which there were other unusual run conditions.}
%  \begin{tabular}{||l|r r r r ||}

\caption[List of ``golden" runs from the 2016 HPS physics run.]{List of ``golden" runs from the 2016 HPS physics run used in this analysis, corrected for livetime, excluding files with DAQ errors, and periods where the SVT bias was off and other unusual run conditions.  An asterisk next to a run number indicates that some of the files in the run are excluded from the list.  In this case, the numbers of events and luminosity listed are of the remaining files in the run.} \\
\label{tbl:runs_used}
 %\hline 
  \hline
  Run Number & Total Events  &	Luminosity (nb$^{-1}$) \\
  \hline
\endfirsthead
\caption{List of ``golden" runs from the 2016 HPS physics run. (continued)}
  %\hline
  \hline
  Run Number & Total Events  &	Luminosity (nb$^{-1}$) \\
  \hline
\endhead
\hline


\endfoot

7629  &  48,445,040  &  37.44\\
7630  &  60,975,330  &  46.48\\
7636  &  148,219,610  &  111.61\\
7637  &  12,100,110  &  9.47\\
7644  &  150,288,740  &  122.42\\
7653  &  25,128,650  &  19.42\\
7779*  &  131,186,351  &  88.81\\
7780  &  121,447,750  &  69.43\\
7781  &  151,580,510  &  92.22\\
7782  &  19,410,430  &  11.11\\
7783  &  3,677,110  &  2.90\\
7786  &  4,746,870  &  2.59\\
7795*  &  112,039,662  &  76.45\\
7796  &  150,763,230  &  86.26\\
7798  &  167,693,140  &  113.94\\
7799*  &  152,364,440  &  104.70\\
7800  &  159,933,840  &  108.51\\
7801*  &  152,350,620  &  84.37\\
7803*  &  157,048,450  &  90.70\\
7804  &  150,163,340  &  97.96\\
7805*  &  152,117,190  &  84.48\\
7807*  &  120,426,420  &  72.37\\
7947  &  100,003,560  &  107.27\\
7948  &  112,391,630  &  121.36\\
7949  &  105,624,080  &  116.90\\
7953  &  25,034,370  &  21.35\\
7962*  &  23,444,541  &  24.52\\
7963  &  100,690,930  &  109.08\\
7964  &  100,426,760  &  103.67\\
7965  &  47,883,630  &  51.25\\
7966  &  102,294,530  &  104.05\\
7968  &  100,021,800  &  97.10\\
7969  &  9,593,000  &  10.02\\
7970  &  100,430,650  &  93.80\\
7972  &  72,335,630  &  82.01\\
7976  &  25,210,890  &  28.57\\
7982  &  16,805,500  &  19.70\\
7983  &  100,237,730  &  115.49\\
7984  &  105,389,430  &  122.87\\
7985  &  103,263,260  &  120.52\\
7986  &  102,740,620  &  120.07\\
7987  &  104,291,800  &  122.51\\
7988  &  100,041,960  &  114.68\\
8025  &  100,257,350  &  112.84\\
8026  &  100,229,880  &  114.00\\
8027  &  103,477,890  &  115.85\\
8028  &  119,665,800  &  132.01\\
8029  &  100,850,170  &  111.75\\
8030  &  68,263,790  &  75.75\\
8031  &  58,215,590  &  64.57\\
8039*  &  100,159,970  &  107.32\\
8040  &  100,283,730  &  114.15\\
8041  &  29,615,580  &  33.61\\
8043*  &  28,089,049  &  32.17\\
8044  &  100,089,230  &  115.04\\
8045  &  101,535,140  &  104.51\\
8046  &  101,280,500  &  99.83\\
8047  &  100,918,360  &  109.15\\
8048  &  100,013,000  &  113.79\\
8049  &  22,101,030  &  24.83\\
8051  &  29,492,890  &  32.59\\
8055  &  54,455,460  &  65.89\\
8057  &  100,049,810  &  120.42\\
8058  &  100,069,290  &  115.98\\
8059  &  110,092,750  &  127.94\\
8072  &  108,117,590  &  131.55\\
8073  &  103,940,210  &  109.11\\
8074  &  88,071,400  &  102.49\\
8075  &  34,367,160  &  42.05\\
8077  &  57,189,610  &  69.26\\
8085  &  59,817,680  &  80.03\\
8086  &  97,369,240  &  123.91\\
8087  &  109,983,980  &  119.70\\
8088  &  27,287,810  &  35.38\\
8090  &  31,698,590  &  39.12\\
8092  &  99,450,940  &  120.41\\
8094  &  100,575,040  &  123.66\\
8095  &  105,290,240  &  106.32\\
8096  &  100,177,890  &  106.45\\
8097  &  99,131,050  &  112.06\\
8098  &  101,838,720  &  126.29\\
8099  &  128,774,050  &  157.58\\

\hline
Total &7,292,550,593 & 7089.78 \\
  \hline
  \hline
%  \end{tabular}

%\end{center}
\end{longtable}  

\begin{table}[htp]
\caption{List of ranges of files excluded from golden runs.  Reasons for excluding regions are explained in more detail in Section \ref{sec:exclude}.  An asterisk indicates that files are missing from tape and are not part of a contiguous range.}
\begin{center}
\begin{tabular}{l l l}
Run Number \# & Excluded Range & Reason for Exclusion \\
\hline
7779 & 0 - 58  & SVT position not at 0.5 mm\\
7795 & 135 - 150 & beam blocker not in place \\
7799 & * & 3 files missing \\
7801 & * & 16 files missing \\
7803 & * & 14 files missing \\
7805 & * & 25 files missing \\
7807 & * & 12 files missing \\
7962 & 0-4 & livetime from mya \\
8039 & 0-11 & livetime from mya \\
8043 & 0-165 & livetime from mya \\

\end{tabular}
\end{center}
\label{tbl:excluded_file_ranges}
\end{table}%

%\section{Salvageable Runs Not Included in Golden Run list}
%There were several runs with 200 nA beam that I did not include in the golden runs list in the next section which have certain problems that may only affect a small fraction of the files in the run.  I could eventually include these runs after determining which file ranges are good and which ones are bad. 
%
%These runs are:
%
%\begin{table}[htp]
%\caption{200 nA runs with some problems in them}
%\begin{center}
%\begin{tabular}{|c|l|}
%run \# & problem \\
%\hline
%7779 & SVT out at start of run.  \\
%7795 & beam blocker removed for two minutes during run \\
%7799 & a few files missing \\
%7801 & ``      " \\
%7803 & ``      " \\
%7805 & ``      " \\
%7807 & ``      " \\
%7962 & Livetime in mya is -999 for first two minutes of run \\
%8043 & Livetime in mya is 0.326 for first two minutes of run \\
%
%\end{tabular}
%\end{center}
%\label{tab:salvageable}
%\end{table}%
%
%
%
%These 9 salvageable 200 nA runs with minor problems account for 5.737 mC of charge (4.965 mC of which has SVT bias on), and have 1459 million events.  
%
%Additionally, we can add the following runs that we took prior to the change from 150 nA to 200 nA.  
%
%7629, 7630, 7636, 7637, 7644, 7653
%
%The total gated beam charge of these 6  runs with 150 nA beam is 2.17 mC, with 445 million total events.  
%
%
%
%
%
%\section{Golden Runs (first pass)}
%I found none of the aforementioned problems in the following runs which have a 200 nA beam:
%
%7780, 7781, 7782, 7783, 7786, 7796, 7798, 7800, 7804, 7947, 7948, 7949, 7953, 7963, 7964, 7965, 7966, 7968, 7969, 7970, 7972, 7976, 7982, 7983, 7984, 7985, 7986, 7987, 7988, 8025, 8026, 8027, 8028, 8029, 8030, 8031, 8040, 8041, 8044, 8045, 8046, 8047, 8048, 8049, 8051, 8055, 8057, 8058, 8059, 8072, 8073, 8074, 8075, 8077, 8085, 8086, 8087, 8088, 8090, 8092, 8094, 8095, 8096, 8097, 8098, 8099
%
%This list of 66 runs encompasses 39.172 mC of beam charge (37.366 mC of which has bias on), and 5.718 billion events.  
%
%By expanding our golden run selection to include the salvageable 200 nA and 150 nA runs, we can increase the number of events available to use by 33\%, and the beam charge by 20\%.  I would consider this to be a worthwhile improvement on the amount of reach we could get in a bump hunt or a vertexing analysis.  


%\section{Some results}
%Figure \ref{fig:events_per_charge} shows the number of events per nC in each of the chosen golden runs and ``salvageable" runs as a function of run number.  This number seems to be consistent during the 2016 runs, with abrupt changes occurring only when the current and/or trigger conditions are changed.  
\section{Conclusions}
This list contains 82 runs, with a total of 7,292,550,593 events and 7089.78 nb$^{-1}$ of luminosity which is equivalent to 2.56 PAC days.  


\end{document}