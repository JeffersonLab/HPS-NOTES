\documentclass[12pt]{article}
%\documentclass[12pt]{report}
\usepackage{graphicx}
\usepackage{epsfig}
\begin{document}

\begin{center}
{\LARGE\bf Procedures for recovering beam from a trip v0.1} 
\end{center}

\begin{center}
{Takashi Maruyama}
\end{center}

\begin{center}
{October 17, 2014}
\end{center}

\section{Contacts}

\noindent
Hall-B beam line expert:

\noindent
Stepan Stepanyan: 757-269-7196 (Office) 757-303-0499 (Cell)

\section{\bf Run Conditions}

After bringing the beam initially, we stay in the following safe run conditions until we understand the beam characteristics well.

\vskip 0.2 in
\noindent
{\bf Safe Conditions:}

\begin{itemize}
\item
Protection collimator with 3 mm gap is in.
\item
Target is in.
\item
SVT layers 1-3 is in retracted-position: L1 at 4.5 mm, L2 at 5.0 mm, L3 at 5.5 mm.
\item
SVT LV is on and bias voltage is at 60 V.
\item
ECal is up.
\item
Beam current is at least 50 nA so that BPM can read.
\end{itemize}

Beam trip will turn off SVT LV and bias voltages.

\section{Procedures for recovering beam from a trip}

\begin{enumerate}
\item
Accelerator operations correct problem and using standard procedures bring beam down to Hall B dump through HPS. Ramp up beam current slowly.
\item
Monitor beam recovery after the trip: BPM�s and halo monitors. 
\item
Check ECal rates and hit pattern.
\item
Turn on SVT LV and bias voltage to 60 V.
\end{enumerate}

If above studies during the initial data taking period show that the beam recovery is benign, change the safe operating conditions.

\begin{itemize}
\item
Increase SVT bias voltage to 200 V.
\item
Increase beam current incrementally to 200 nA.
\item 
Move SVT layers 1-3 closer to the beam.
\end{itemize}

\end{document}

