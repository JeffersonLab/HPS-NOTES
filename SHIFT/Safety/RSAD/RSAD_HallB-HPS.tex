\documentclass [12pt]{article}

\title{Radiological Safety Analysis Document for Heavy Photon Search experiment engineering run}

\begin{document}
\date{}
\vskip 0.5cm
\maketitle

\noindent

{\bf This Radiological Safety Analysis Document (RSAD) will identify the
general conditions associated with running the Heavy Photon Search experiment in Hall B, and
controls with regard to production, movement, or import of radioactive
materials. }\footnote{Contact person: Stepan Stepanyan}

\section{Description}
\indent

The engineering run of Heavy Photon Search (HPS) experiment, E12-11-006, will take place from October 2014 to June 2015 in the experimental Hall-B.  
The HPS run will use a detector located in the downstream alcove of Hall-B. The setup is based on a three magnet chicane where the first and the last dipoles serve as bending magnets, while the middle one, 18D36 dipole, will be used as a spectrometer magnet. The beam will be transported through the hall to the HPS target using the standard $3$ inch vacuum beam pipe. There are no other targets or vacuum windows along the beam before and after the HPS target. The target and the tracking detector are located in the vacuum. A set of vacuum chambers will allow passage of beam to the dump in the vacuum. The vacuum beam line and the chicane magnets are configured in such a way that the beam will have clear passage to the Hall B electron beam dump whether or not the chicane magnets are energized.. Chicane magnet power supplies are interlocked with beam delivery system (FSD). Beam delivery will be terminated in the event of a magnet power supply trip. 

For production data taking HPS will use a $4$ $\mu$m thick (0.125\% r.l.) tungsten target located at the beginning of the analyzing magnet and up to $400$ nA electron beams at energies $1.1$ GeV and $2.2$ GeV. There will be other target foils mounted on the target ladder, $500$ $\mu$m carbon ($0.3\%$ r.l.), $700$ $\mu$m CH$_2$ ($0.25\%$ r.l.), and $8$ $\mu$m tungsten ($0.25\%$ r.l.) for calibration purposes. The maximum luminosity during the run will be $< 10^{34}$ sec$^{-1}$ cm$^{-2}$ per nucleon. During running with beam currents above $50$ nA, Hall B beam stopper (a $30$ cm long cooled copper absorber) will be positioned before the Faraday Cup, to prevent overheating. The bremsstrahlung photons produced in the target will have $30$ mrad angle relative to the electron beam and will be directed to a photon beam dump mounted behind the first shielding wall between the Hall-B downstream alcove and the tunnel. The photon dump will be the same as during the CLAS TPE experiment, although the HPS photon beam intensity will be $100$ times smaller. 

The HPS detectors will be located in close proximity of the electron beam plane. Detail GEANT simulations have been carried out to asses possible radiation effects to the detectors and electronics (see for example effect of the neutron radiation on electronics described in Appendix).


\section{Summary and Conclusions}
\indent

The experiment is not expected to produce significant levels of
radiation at the site boundary. However, it will be periodically
monitored by the Radiation Control Department to ensure that the site
boundary goal is not exceeded. The main consideration is the
manipulation and/or handling of target(s) or beam line hardware. As
specified in Sections IV (B) and VII, the manipulation and/or handling
of target(s) or beam line hardware (potential radioactive material),
the transfer of radioactive material, or modifications to the beam line
after the target assembly must be reviewed and approved by the
Radiation Control Departement. 

Adherence to this RSAD is vital.


\section{ Calculations of Radiation Deposited in the Experimental Hall
(the Experiment Operations Envelope)} 
\indent

The radiation budget for a given experiment is the amount of radiation
that is expected at site boundary as a result of a given set of
experiments. This budget may be specified in terms of mrem at site
boundary or as a percentage of the Jefferson Lab design goal for dose
to the public, which is 10 mrem per year. The Jefferson Lab design
goal is 10\% of the DOE annual dose limit to the public, and cannot 
be exceeded without prior written consent from the Radiation Control
Department Head, the Director of Jefferson Lab, and the Department of Energy. 

Calculations of the contribution to Jefferson Lab's annual radiation
budget that would result from running under a broad variety of
conditions typical of Hall B operations indicate that the contribution
from this experiment will be negligible. With this expectation, we
have not carried out calculations for the specific running conditions
of this experimental group. 

This expectation will be verified during the experiment by using the
active monitors at the Jefferson Lab site boundary to keep up with the
dose for the individual setups from Hall B and the other Halls. If it
appears that the radiation budget will be exceeded, the Radiation
Control Department (RCD) will require a meeting with the experimenters and
the Head of the Physics Division to determine if the experimental
conditions are accurate, and to assess what actions may reduce the
dose rates at site boundary. If the site boundary dose approaches or
exceeds 10 mrem during any calendar year, the experimental program
will stop until a resolution can be reached. 

\section{Radiation Hazards}
\indent

The following controls shall be used to prevent the unnecessary
exposure of personnel and to comply with Federal, State, and local
regulations, as well as with Jefferson Lab and the Experimenter's home
institution policies. 

\subsection{From Beam in the Hall}

When the Hall status is Beam Permit, there are potentially lethal
conditions present. Therefore, prior to going to Beam Permit, several
actions will occur. Announcements will be made over the intercom
system notifying personnel of a change in status from Restricted
Access (free access to the Hall is allowed, with appropriate dosimetry
and training) to Sweep Mode. All magnetic locks on exit doors will be
activated. Persons trained to sweep the area will enter by keyed
access (Controlled Access) and search in all areas of the Hall to
check for personnel. 

After the sweep, another announcement will be made, indicating a
change to Power Permit, followed by Beam Permit. The lights will dim
and Run-Safe boxes will indicate "OPERATIONAL" and "UNSAFE". IF YOU ARE IN THE HALL AT ANY TIME THAT THE RUN-SAFE BOXES INDICATE �UNSAFE�, IMMEDIATELY HIT THE BUTTON ON THE BOX. 

Controlled Area Radiation Monitors (CARMs) are located in strategic
areas around the Hall and the Counting House to ensure that unsafe
conditions do not occur in occupiable areas. 

\subsection{From Activation of Target and Beam line Components}

All radioactive materials brought to Jefferson Lab shall be identified
to the Radiation Control Department. These materials include, but are not
limited to radioactive check sources (of any activity, exempt or
non-exempt), previously used targets or radioactive beam line
components, or previously used shielding or collimators. The RCD
inventories and tracks all radioactive materials onsite. The Radiation
Control Department will survey all experimental setups before experiments
begin as a baseline for future measurements. 

The Radiation Control Department will coordinate all movement of used
targets, collimators, and shields. The Radiation Control Department will
assess the radiation exposure conditions and will implement controls
as necessary based on the radiological hazards. 

There shall be no local movement of activated target configurations
without direct supervision by the Radiation Control Department. Remote
movement of target configurations shall be permitted, providing the
method of movement has been reviewed and approved by the Radiation
Control Department. 

No work is to be performed on beam line components, which could result
in dispersal of radioactive material (e.g., drilling, cutting,
welding, etc.). Such activities must be conducted only with specific
permission and control of the Radiation Control Department. 

\section {Incremental Shielding or Other Measures to be Taken to
Reduce Radiation Hazards} 

None.


\section {Operations Procedures}

All experimenters must comply with experiment-specific administrative
controls. These controls begin with the measures outlined in the
experiment's Conduct of Operations Document, and also include, but are
not limited to, Radiation Work Permits, Temporary Operational Safety
Procedures, and Operational Safety Procedures, or any verbal
instructions from the Radiation Control Department. A general access RWP is
in place that governs access to Hall B and the accelerator enclosure,
which may be found in the Machine Control Center (MCC); it must be
read and signed by all participants in the experiment. Any individual
with a need to handle radioactive material at Jefferson Lab shall
first complete Radiation Worker (RW I) training.


There shall be adequate communication between the experimenter(s) and
the Accelerator Crew Chief and/or Program Deputy to ensure that all
power restrictions on the target are well known. Exceeding these power
restrictions may lead to excessive and unnecessary contamination,
activation, and personnel exposure. 

No scattering chamber or downstream component may be altered outside
the scope of this RSAD without formal Radiation Control Department
review. Alteration of these components (including the exit beam line
itself) may result in increased radiation production from the Hall and
a resultant increase in site boundary dose. 

\section {Decommissioning and Decontamination of Radioactive Components}

Experimenters shall retain all targets and experimental equipment
brought to Jefferson Lab for temporary use during the
experiment. After sufficient decay of the radioactive target
configurations, they shall be delivered to the experimenter's home
institution for final disposition. All transportation shall be done in
accordance with United States Department of Transportation Regulations
(Title 49, Code of Federal Regulations) or International Air Transport
Association regulations. In the event that the experimenter's home
institution cannot accept the radioactive material due to licensing
requirements, the experimenter shall arrange for appropriate funds
transfers for disposal of the material. Jefferson Lab cannot store
indefinitely any radioactive targets or experimental equipment. 

{\bf The Radiation Control Department may be reached at any time through the
Accelerator Crew Chief (269-7050). }

\vspace{3.cm}
Approvals:

\vspace{1.cm}
\line(1,0){190}                    
\hspace{3cm}         \line(1,0){90}  

Radiation Control Department Head               \hspace{3cm}                  Date


\clearpage
\section{Appendix: Radiation damage to the HPS electronics}

While most of the HPS electronics are located away from the beam line, the front-end electronics 
boards (FE boards) are inside the vacuum chamber of the PS magnet at 20 cm from the HPS target. 
There are ten FE boards and each FE board has one Xilinx Artix 100T FPGA and 25 voltage regulator
MOSFETs, which are potentially vulnerable to radiations, especially neutrons. While the radiation 
damage from Total Ionizing Dose (TID) and Non Ionizing Energy Loss (NIEL) are negligible, Single
Event Upset (SEU) may cause data corruption in FPGA and Single Event Gate Rupture (SEGR) may
permanently damage MOSFET. 

BaBar detector at SLAC had 47 FPGAs near the interaction point and observed many SEUs. Based on
the BaBar experience and taking the geometry difference into consideration, we estimate 1 SEU/day
at the neutron flux of 8000 neutrons/cm$^2$/sec.

Since all the MOSFETs are operated at less than 5 V gate voltage, SEGR is verly unlikely to take 
place.

\subsection{Neutron production calculation}

Neutron production calculations were made using the particle interaction simulation code FLUKA 
and Geant4. Following neutron production sources have been studied,

\begin{enumerate}
\item
SVT Protection Collimator (1 cm Tungsten) 3 m upstream of the HPS target

\item
HPS Target (4 $\mu$m Tungsten)

\item
PS Magnet Coil/Body at the beam's right

\item
ECal vacuum chamber and downstream beam pipe

\item
Beam dump
\end{enumerate}

Among these sources, the HPS target is by far the dominant neutron source for the FE boards and
the neutron production rate is 2$\times$10$^{-6}$ neutrons per 2.2 GeV e-. The neutron flux at 
the FE boards is estimated to be 700 neutrons/cm$^2$/sec at the beam current of 200 nA. 
This neutron flux is at least a factor of ten lower than the neutron flux expected to cause 
1 SEU/day.    

\subsection{Radiation damage to the standard electronics components in the Hall}

The experiment is not expected to produce significant levels of neutron 
radiation that may cause damage to the Hall electronics located on the Forward Carriage or on the Space Frame. The beam energies and the integrated luminosity of the experiment
are in the same range as for the CLAS nuclear target experiments. Furthermore, the target and the detector are located in the downstream alcove. Any radiation produced on the target or in any parts of the detector will be confined to the alcove or downstream tunnel (towards beam dump).


\end{document}

