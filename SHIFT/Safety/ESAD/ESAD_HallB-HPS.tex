\documentclass[11pt]{report}
\usepackage{tabularx,graphicx,layout}
\usepackage{hyperref}
\usepackage{blindtext}
\usepackage[utf8]{inputenc}
\usepackage{enumitem}
\usepackage{pifont}

\begin{document}

\title{ Experiment Safety Assessment Document (ESAD) \\ 
Heavy Photon Search experiment engineering run}
%
% LaTeX Version 12 GeV ESAD 
%
% Update 7 March 2014
%
\maketitle
\tableofcontents

\chapter{Introduction}

This ESAD document describes identified hazards of the HPS experiment and the measures taken to eliminate, control, or mitigate them.
This document is part of the CEBAF experiment review process as defined in
\href{http://www.jlab.org/ehs/ehsmanual/manual/3120.html}{Chapter 3120 of the Jefferson Lab EHS\&Q manual},
and will start by describing general types of hazards that might be present in any of the  
JLab experimental halls.  The document then addresses the hazards associated 
with all the experimental sub-systems of Experimental Hall B and HPS and their 
mitigation.  Responsible personnel for each item is also noted.  
In case of life threatening 
emergencies call 911 and then notify the guard house at 5822 so that the guards can help
the responders.  This document does not attempt to describe the function 
or operation of the various sub-systems. Such information can be found in
the experimental hall specific Operating Manuals.

%{\it{{\bf{TO DO LIST}}
%\begin{itemize}
%\item update the outline in Chapter 3120 to match our document final ESAD
%\item review as well as remove
%the es\&h coordinators as the physics division liason does those tasks
%\item use "responsible personnel' notation through-out
%\item we can add a reference list of names at the end; but for now I have them all with
%      the sections they go with
%\end{itemize}
%}}

\chapter{General Hazards}

\section{Radiation}
	
CEBAF's high intensity and high energy electron beam is a potentially lethal direct radiation source. 
It can also create radioactive materials that are hazardous even  after the beam has been turned off. 
There are many redundant measures aimed at preventing accidental exposure to personnel by the beam 
or exposure to beam-associated radiation sources that are in place at JLab. The training and mitigation 
procedures are handled through the JLab Radiation Control Department (RadCon). The radiation safety 
department at JLab can be contacted as follows: For routine support and surveys, or for emergencies 
after-hours, call the RadCon Cell phone at 876-1743. For escalation of effort, or for emergencies, 
the RadCon manager (Vashek Vylet) can be reached as follows: Office: 269-7551, Cell: 218-2733 or Home: 772-6098.

Radiation damage to materials and electronics is mainly determined by the neutron 
dose (photon dose typically causes parity errors and it is easier to shield against). 
Commercial-off-the-shelf (COTS) electronics is typically robust up to neutron 
doses of about $10^{13} n/cm^2$. If the experimental equipment dose as calculated 
in the RSAD is beyond this damage threshold, the experiment needs to add 
an appendix on "Evaluation of potential radiation damage" in the experiment 
specific ESAD. There, the radiation damage dose, potential impact to equipment 
located in areas above this damage threshold as well as mitigating measures taken should be described.

\section{Fire}

	The experimental halls contain numerous combustible materials and flammable gases. 
In addition, they contain potential ignition sources, such as electrical wiring and equipment. 
General fire hazards and procedures for dealing with these are covered by JLab emergency 
management procedures. The JLab fire protection manager (Dave Kausch) can be contacted at 269-7674.

\section{Electrical Systems}

	Hazards associated with electrical systems are the most common risk in the experimental halls. 
Almost every sub-system requires AC and/or DC power. Due to the high current and/or high voltage 
requirements of many of these sub-systems they and their power supplies are potentially lethal 
electrical sources. In the case of superconducting magnets the stored energy is so large that 
an uncontrolled electrical discharge can be lethal for a period of time even after the actual 
power source has been turned off.  Anyone working on electrical power in the experimental Halls 
must comply with \href{http://www.jlab.org/ehs/ehsmanual/manual/6200.html}{Chapter 6200 of the Jefferson Lab EHS\&Q manual}
and must obtain approval of one of the responsible personnel. 
The JLab electrical safety point-of-contact (Todd Kujawa) can be reached at 269-7006.

\section{Mechanical Systems}

	There exist a variety of mechanical hazards in all experimental halls at JLab. 
Numerous electro-mechanical sub-systems are massive enough to produce potential fall 
and/or crush hazards.  In addition, heavy objects are routinely moved around within 
the experimental halls during reconfigurations for specific experiments. 

Use of ladders and scaffold must comply 
with \href{http://www.jlab.org/ehs/ehsmanual/manual/6132.html}{Chapter 6231 of the 
Jefferson Lab EHS\&Q manual}.
Use of cranes, hoists, lifts, etc. must comply with
\href{http://www.jlab.org/ehs/ehsmanual/manual/6141.html}{Chapter 6141 of the 
Jefferson Lab EHS\&Q manual}. 
Use of personal protective equipment 
to mitigate mechanical hazards, such as hard hats, safety harnesses, and safety 
shoes are mandatory when deemed necessary.
The JLab technical point-of-contact (Suresh Chandra) can be contacted at 269-7248.

\section{Strong Magnetic Fields}

	Powerful magnets exist in all JLab experimental halls. Metal objects may be attracted 
by the magnet fringe field, and become airborne, possibly injuring body parts or striking 
fragile components resulting in a cascading hazard condition. Cardiac pacemakers or other 
electronic medical devices may no longer function properly in the presence of magnetic fields. 
Metallic medical implants (non-electronic) may be adversely affected by magnetic fields. Loss of 
information from magnetic data storage devices such as tapes, disks, and credit cards may also occur. 
Contact Jennifer Williams at 269-7882, in case of questions or concerns.

\section{Cryogenic Fluids and Oxygen Deficiency Hazard}

	{\bf Not applicable for this experiment.}

\section{Vacuum and Pressure Vessels}

	Vacuum and/or pressure vessels are commonly used in the experimental halls. Many 
of these have thin Aluminum or kevlar/mylar windows that are close to the entrance 
and/or exit of the vessels or beam pipes. These windows burst if punctured accidentally 
or can fail if significant over-pressure were to exist. Injury is possible if a failure 
were to occur near an individual. All work on vacuum windows in the experimental halls 
must occur under the supervision of appropriately trained JLab personnel. Specifically, 
the scattering chamber and beam line exit windows must always be leak checked before service. 
Contact Will Oren 269-7344 for vacuum and pressure vessel issues.

\section{Hazardous Materials}

	Hazardous materials in the form of solids, liquids, and gases that may harm people 
or property exist in the JLab experimental halls. The most common of these materials include 
lead, beryllium compounds, and various toxic and corrosive chemicals. 
Material Safety Data Sheets (MSDS) for hazardous materials 
in use in the Hall are available from the Hall safety warden.  These are being replaced by the new standard
Safety Data Sheets (SDS) as they become available in compliance with the new OSHA standards.    Handling of these materials 
must follow the guidelines of the EH\&S manual. Machining of lead or beryllia, that 
are highly toxic in powdered form, requires prior approval of the EH\&S staff. 
Lead Worker training is required in order to handle lead in the Hall. 
In case of questions or concerns, the JLab hazardous materials specialist (Jennifer Williams) can be contacted at 269-7882.

\section{Lasers}

	{\bf Not applicable for this experiment.}
%
% details for each of the Halls
%
\chapter{Hall Specific Equipment}

\section{Overview}
\indent

The following Hall B subsystems are considered part of the experimental endstation equipment for running the Heavy Photon Search (HPS) experiment engineering run.
Many of these subsystems impose similar hazards, such as those induced by magnets and magnet power supplies,
high voltage systems and vacuum systems.  Note that a specific sub-system may have many unique hazards associated with it.
For each major system, the hazards, mitigations, and responsible personnel are noted.

The material in this chapter is a subset of the material in the full Hall B operations manual and is only intended to familiarize
people with the hazards and responsible personnel for these systems.  It in no way should be taken as sufficient information to use or operate this equipment.

\section{Checking Tie-in To Machine Fast Shutdown System}
\indent

In order to make sure that hall equipment that should be tied into the machine fast shutdown (FSD) system
has been properly checked, the hall work coordinator must be notified by e-mail prior to the end of each
installation period by the system owner
that the checks have been performed in conjunction with accelerator operations (i.e. checking that equipment's signals
will in fact cause an FSD).  These notifications will be
noted in the work coordinator's final check-list as having been done.   System owners are responsible
for notifying the work coordinator that their system has an FSD tie-in so it can be added to the check-list.

\section{Beamline}
\indent

The control and measurement equipment along the Hall B beamline consists of various elements necessary to transport beam with required specifications onto the production target and the beam dump, and simultaneously measure the properties of the beam relevant to the successful implementation of the physics program in Hall B. 

The beamline in the Hall provides the interface between the CEBAF accelerator and the experimental hall. All work on the beamline must be coordinated with both physics division and accelerator division in order to ensure safe and reliable transport of the electron beam to the dump.

\subsection{Hazards} 

Along the beamline various hazards can be found. These include radiation areas, vacuum windows, high voltage, and magnetic fields.

\subsection{Mitigations}

\indent

All magnets (dipoles, quadruples, sextuples, beam correctors) and beam diagnostic devices (BPMs, scanners, beam loss monitors, viewers) necessary to transport and monitor the beam are controlled by Machine Control Center (MCC) through EPICS \cite{epics}, except for specific elements which are addressed in the subsequent sections. The detailed safety operational procedures for the Hall B beamline should be essentially the same as the one for the CEBAF machine and beamline.

Personnel who need to work near or around the beamline should keep in mind the potential hazards:
\begin{itemize}
\item Radiation "Hot Spots" - marked by ARM of RadCon personnel,
\item Vacuum in beamline tubes and other vessels,
\item Thin windowed vacuum enclosures (e.g. the scattering chamber),
\item Electric power hazards in the vicinity of magnets, and 
\item Conventional hazards (fall hazard, crane hazard, etc.). 
\end{itemize} 

These hazards are noted by signs and the most hazardous areas along the beamline are roped off to restrict access when operational (e.g. around the HPS chicane magnets). Signs are posted by RadCon for any hot spots. Survey of the beamline and around it will be performed before work is done on the beam line or around. The connection of leads to magnets have plastic covers for electrical safety. Any work around the magnets will require de-energizing the magnets. Energized magnets are noted by read flashing beacons. Any work on the magnets requires the "Lock and Tag" procedures \cite{ehs}.  

Additional safety information can be obtained from the following documents:
\begin{itemize}[label=$\circ$]

\item EH\&S Manual \cite{ehs} 

\item PSS description Document \cite{pss}

\item Accelerator Operations Directive \cite{ops}

\end{itemize} 

\subsection{Responsible personnel}

The beamline requires both accelerator and physics personnel to maintain and operate. It is very important that both groups stay in contact with each other to coordinate any work on the Hall B beamline.  

 \begin{table}[!ht]
 \centering
 \begin{tabular}{|c|c|c|c|c|}
\hline
 Name&Dept.&Phone&email&Comments \\ \hline
  F-X. Girod & Hall-B&x6002&fxgirod@jlab.org& 1st contact  \\ \hline
 S. Stepanyan & Hall-B&x7196&\href{mailto:stepanyan@jlab.org}{\nolinkurl{stepanyan@jlab.org}}&2nd contact \\ \hline
  M. Tiefenback &Accel.&x7430&\href{mailto:tiefen@jlab.org}{\nolinkurl{tiefen@jlab.org}}& Contact to Hall-B \\ \hline
% H. Areti& Accel.&&\href{mailto:areti@jlab.org}{\nolinkurl{areti@jlab.org}}&Contact to Physics \\ \hline
\end{tabular}
\caption{Responsible personnel for the Hall B beamline.} 
\label{tb:beam}
\end{table}


\section{HPS Chicane Magnets}
\indent

The HPS experiment will use a three magnet chicane installed on the beamline in the Hall B downstream alcove. The chicane includes the Hall B pair spectrometer magnet, an 18D36 dipole, with its vacuum chamber, and two identical H-dipoles known as "Frascati" magnets. Electron beam will pass through the magnets in the vacuum. The pair spectrometer magnet will serve as a spectrometer magnet for HPS. The two "Frascati" dipoles are to keep the deflected electron beam on the beamline to the dump. The spectrometer magnet will be powered from the Hall B pair spectrometer magnet power supply, DANFYSIK-8000. The "Frascati" magnets will be powered from so called "mini-torus" power supply, Dynapower PD42-04000103-GKLX-PY57.   
 
\subsection{Hazards} 
\indent

Dipole magnets can present magnetic and electrical hazards when they are energized. There is also a possible hazard of unwanted beam motion if there is a magnet power trip.  


\subsection{Mitigations}
\indent

There are plastic covers on the connection panels for power leads on the magnets for electrical safety. Any work around the magnets will require de-energizing the magnets. Energized magnets are noted by red flashing beacons. Any work on the magnets requires a "Lock and Tag" procedure \cite{ehs}. There will be beacons installed to notify when magnetic field is present. The magnet power supplies will be interlocked to the beam shutdown system (FSD). If any of power supplies will trip, beam delivery to Hall B will be interrupted.

\subsection{Responsible personnel}

The chicane magnets will be maintained by the Hall B engineering group.  

 \begin{table}[!ht]
 \centering
 \begin{tabular}{|c|c|c|c|c|}
\hline
 Name&Dept.&Phone&email&Comments \\ \hline
 Engineering-on-call & Hall-B&(757) 584-5245&$-$& 1st contact  \\ \hline
 D. Tilles & Hall-B&(757) 810-9576&\href{mailto:tilles@jlab.org}{\nolinkurl{tilles@jlab.org}}&2nd contact \\ \hline
  \end{tabular}
\caption{Personnel responsible for the HPS chicane.} 
\label{tb:magnets}
\end{table}


\section{Target System}

\indent

The HPS target system consists of several solid material foils mounted on a target ladder. The bottom edge of the foils mounted on the ladder is free-standing so there
is no thick support frame to trip the beam when the target is inserted. Target position is remotely  adjustable vertically allowing different targets to be inserted. The support
frame on the beam-right side of the target is made thin enough to
prevent excessive flux of secondaries and radiation damage to the silicon detector in the event of an errant
beam,  caused, for example, by an chicane magnet trip which will move beam to the right.
               
\subsection{Hazards} 

\indent

There are hazards related to moving the target frame into the beam or overheating the target foils. The stepping motor linear
actuator will be operated using EPICS controls.  The GUI for operation of the target will have preset coordinates for each target foil.  
The tungsten targets are intended to operate with beam currents up to $500$ nA, which produce strong local heating. The strength of tungsten drops by an order of magnitude with temperature increases in the range of 1000 C. In addition, the material re-crystallizes above
this range, which increases the tendency for cracking where thermal expansion has caused temporary dimpling. 

\subsection{Mitigations}

\indent

There will be limit switches (hard stops) that will prevent the motion of the target ladder outside of allowed range if EPICS set values are wrong. To keep the temperature rise less than about 1000 degrees, we adjust the optics to produce an adequately large beam spot and limit the maximum current incident on the target. There will be overall beam current limit of $500$ nA for the experiment.

\subsection{Responsible personnel}

\indent

The target system will be maintained by the Hall B engineering group.  

 \begin{table}[!htb]
 \centering
 \begin{tabular}{|c|c|c|c|c|}
\hline
 Name&Dept.&Phone&email&Comments \\ \hline
 Engineering-on-call & Hall-B&(757) 584-5245&$-$& 1st contact  \\ \hline
 D. Tilles & Hall-B&(757) 810-9576&\href{mailto:tilles@jlab.org}{\nolinkurl{tilles@jlab.org}}&2nd contact \\ \hline
C. Field& SLAC&$-$&\href{mailto:sargon@slac.stanford.edu}{\nolinkurl{sargon@slac.stanford.edu}}&contact \\ \hline
 \end{tabular}
\caption{ Personnel responsible for the target.} 
\label{tb:target}
\end{table}


\section{Vacuum System}

\indent

The Hall B vacuum system consists of three segments, all interconnected. The beam transport line consisting of $1.5$ to $2.5$ inch beam pipes, the Hall B tagger magnet vacuum chamber, and the set of vacuum chambers through the HPS detector system. Only the tagger vacuum chamber has a large window, $8$ inches over $30$ ft Kevlar-Mylar composite window. There is a $2.5$ in diameter $7$ mil Kapton window at the end of the HPS detector vacuum system, before the shielding wall, that is normally inaccessible. The vacuum in the system is provided by a set of rough, turbo, and ion pumps and it is maintained at the level of better than $10^{-5}$ Torr. 

\subsection{Hazards} 

\indent

Hazards associated with the vacuum system are due to rapid decompression in case of a window failure. Loud noise can cause hearing loss. Also, there is a hazard related to SVT coolant leaking into analyzing magnet vacuum chamber that will degrade the vacuum and may damage readout electronics if the leak is extensive.

\subsection{Mitigations}

\indent

All personnel working in the vicinity of the tagger vacuum chamber window are required to wear ear protection. Warning signs must be posted in that area. To mitigate a possible coolant leakage, the cooling system will be interlocked with the beamline vacuum system and the cooling system pressure gage. In an event of a leak, evident by increased vacuum system pressure or decreased cooling system pressure, the cooling system will be shutdown and valved off. 

\subsection{Responsible personnel}

\indent

The vacuumsystem will be maintained by the Hall B engineering group.  

 \begin{table}[!htb]
 \centering
 \begin{tabular}{|c|c|c|c|c|}
\hline
 Name&Dept.&Phone&email&Comments \\ \hline
 Engineering-on-call & Hall-B&757) 584-5245&$-$& 1st contact  \\ \hline
 D. Tilles & Hall-B&757) 810-9576&\href{mailto:tilles@jlab.org}{\nolinkurl{tilles@jlab.org}}&2nd contact \\ \hline
 \end{tabular}
\caption{Personnel responsible for the vacuum system.} 
\label{tb:vacuum}
\end{table}


\section{Silicon Tracker}
\indent

The silicon vertex tracker (SVT) is a compact six layer tracking system, less than a meter long, which uses silicon microstrip detectors to measure charged particle momentum and decay vertex positions. Each layer, top or bottom, consists of axial and stereo silicon sensors. The SVT  is divided into top and bottom sections  to avoid direct interactions with the beam and degraded electrons, and it resides in vacuum, to eliminate beam gas backgrounds. The first three layers can be moved close to the beam, to maximize acceptance for heavy photons. A cooling system removes heat from the electronics, and cools the sensors to improve their radiation hardness.  The sensors are readout with onboard electronics which pass signals to Front End boards for digitization and transmission out of the vacuum enclosure.

\subsection{Hazards} 
\indent

Hazards to personnel include the high voltage which biases the sensors, and the low current which powers the readout electronics.

Hazards to the SVT itself include mechanical damage, radiation damage, and overheating. Hazards to the vacuum system could arise from excessive SVT outgassing or coolant leaks.
SVT mechanical damage could occur if the top sensors are accidentally driven into the bottom sensors.

Radiation damage could occur in the SVT  if the sensors are driven too close to the beam, the beam moves into the sensors, the beam interacts upstream to produce excessive radiation, or excessive beam currents create more radiation than can be tolerated.

Overheating can occur in the SVT  if the cooling system is performing inadequately or if a cooling system leak develops.

\subsection{Mitigations}
\indent

Hazards to personnel are mitigated by turning off HV and LV power before disconnecting cables or working on the sensors and internal electronics.
Hazards to the Hall B vacuum system have been mitigated by extensive testing of all components to ensure low-outgassing rates, construction of the SVT and electronics in a clean room, and tests of the cooling system to high pressures to prove leak-tightness. The coolant used, water glycol, would not cause irreparable damage to the vacuum system if a leak occurred. If the system pressure  increases, the coolant supply is halted.

Possible mechanical damage has been mitigated by designing the channels which hold the sensors to touch before any modules would touch. Software limits and  limit switches on the motion controllers also prevent the sensors from moving into each other or too close to the beam.

Radiation damage from the beams is mitigated in several steps. First, beam size and halo must conform to beam requirements before beams are passed through the detector. Second, the beams are centered between the top and bottom sections of the SVT. Third, an upstream collimator is aligned with the "centered" beam position to intercept the beam if it moves off nominal position. Fourth, beam halo monitors sense a rise in backgrounds if the beam moves off nominal position, activating the FSD and removing the power permissive to the SVT. Fifth, precision movers position the SVT layers precise and safe distances from the beam . Finally, beam currents and target thicknesses are carefully chosen to avoid over-radiating the silicon sensors.

Overheating is mitigated by requiring good coolant flow, proper coolant temperature, good vacuum (assuring no coolant leakage), and sensor temperature in range  in the interlock for SVT HV and LV power. 

\subsection{Responsible personnel}
\indent

Individuals responsible for the system are:

 \begin{table}[!htb]
 \centering
 \begin{tabular}{|c|c|c|c|c|}
\hline
 Name&Dept.&Phone&email&Comments \\ \hline
Tim Nelson& SLAC&$-$&\href{mailto:tknelson@slac.stanford.edu}{\nolinkurl{tknelson@slac.stanford.edu}}& First contact \\ \hline
Omar Moreno & UCSC & $-$&\href{mailto:omoreno@slac.stanford.edu}{\nolinkurl{omoreno@slac.stanford.edu}}& Contact \\ \hline
Sho Uemura&SLAC&$-$ &\href{mailto:meeg@slac.stanford.edu}{\nolinkurl{meeg@slac.stanford.edu}}& Contact \\ \hline
Per Hansson&SLAC&$-$ &\href{mailto:phansson@slac.stanford.edu}{\nolinkurl{phansson@slac.stanford.edu}}& Contact \\ \hline
 \end{tabular}
\caption{Personnel responsible for the silicon tracker.} 
\label{tb:svt}
\end{table}



\section{Electromagnetic Calorimeter}
\indent

The Electromagnetic Calorimeter (ECal) consists of $442$ lead-tungstate (PbWO$_4$) crystals with avalanche photodiode (APD) 
readout and amplifiers enclosed inside a temperature controlled enclosure. There are two identical ECal modules positioned above and below of the beam plane. In order to operate the calorimeter modules,  high voltage and low voltage are supplied to each channel. The high voltage is $<450$ V and $<1$ mA. The required low voltage is $\pm 5$ V for preamplifier boards. Constant temperature inside the enclosure is kept by running a coolant through the copper pipes that are integrated into the enclosure using laboratory chiller. Cooling system should provide temperature stability at the level of $1^\circ$C.

\subsection{Hazards} 
\indent

Hazards associated with this device are electrical shock or damage to the APDs if the enclosure is opened with the  HV on. There is also hazard associated with coolant leak that may damage preamplifier boards.

\subsection{Mitigations}
\indent

Whenever any work has to be done on the calorimeter, whether it will be opened or not, HV and LV must be turned off. Turn chiller off if enclosure will be opened for maintenance. Any large (more than couple of degrees in C) must be investigated to make sure that there are no leaks.   

\subsection{Responsible personnel}
\indent

Individuals responsible for the system are:

\begin{table}[!htb]
 \centering
 \begin{tabular}{|c|c|c|c|c|}
\hline
 Name&Dept.&Phone&email&Comments \\ \hline
 S. Stepanyan & Hall-B&x7196&\href{mailto:stepanya@jlab.org}{\nolinkurl{stepanya@jlab.org}}&1st contact \\ \hline
 R. Dupre & ORSAY&$-$&\href{mailto:dupre@ipno.in2p3.fr}{\nolinkurl{dupre@ipno.in2p3.fr}}& 2nd contact  \\ \hline
 \end{tabular}
\caption{Personnel responsible for the HPS electromagnetic calorimeter.} 
\label{tb:ecal}
\end{table}

\section{CLAS12 Construction work}
\indent

The main 12 GeV activity in Hall-B during the HPS engineering run will be assembly of the CLAS12 Torus magnet. Beam running would occur during evenings, nights, and weekends or during other periods when it would not conflict with the regularly scheduled assembly of the CLAS12 Torus coils.

\subsection{Hazards} 
\indent

There are no personal hazards associated with the running over evenings, nights, and weekends, and continue with torus assembly during the normal work hours. The only hazard is a possible delay of start of the torus work after beam delivery if there were activation of the beamline close to the torus assembly fixtrues. There may be a possibility that the beam pipe be damaged because of mechanical work around it, and that vacuum could be lost.  
 
\subsection{Mitigations}
\indent

Every time the Hall will switch from beam running to torus assembly, a full radiation survey will be conducted and the Hall will be brought to "Restricted Access". Normally this will happen very early in the morning of work day (6am). If elevated activity near the torus assembly fixtures is found, work on torus must be delayed until conditions are acceptable. 

However, we do not expect any excess radiation in the Hall or activation of any beam line components near the torus assembly area. The HPS  target is located $\sim20$ meters downstream of the assembly area and only tuned electron beam, couple of hundred micron wide, will be transported in vacuum through hall to the target. If beam conditions are not acceptable, which could result in activation near the torus construction, the beam will be tuned until the condition is corrected. Every time beam tune is required the Hall-B tagger magnet will be energized and beam will be dumped on the Hall-B tagger dump, shielded hole in the floor $\sim 15$ meters upstream of the torus assembly area. 

During workday torus assembly work, two vacuum valves installed upstream and downstream of the work area will be closed and vacuum can be bleed in the tours section or beam pipe can be removed if necessary.
 
\subsection{Responsible personnel}
\indent

Individuals responsible for the coordination of the torus assembly and HPS engineering run:

\begin{table}[!htb]
 \centering
 \begin{tabular}{|c|c|c|c|c|}
\hline
 Name&Dept.&Phone&email&Comments \\ \hline
PDL & Hall-B&(757) 876-1789&$-$&1nd contact \\ \hline
S. Stepanyan & Hall-B&x7196&\href{mailto:stepanya@jlab.org}{\nolinkurl{stepanya@jlab.org}}&contact \\ \hline
D. Kashy & Hall-B&(757) 876-3265&\href{mailto:kashy@jlab.org}{\nolinkurl{kashy@jlab.org}}&contact \\ \hline
 \end{tabular}
\caption{Personnel responsible for coordination of the HPS run and the torus assembly.} 
\label{tb:clas12}
\end{table}

\begin{thebibliography}{99}
\bibitem{epics} EPICS Documentation. URL: http://www.epics.org/. See also http://www.aps.anl.gov/asd/controls/epics/EpicsDocumention\\/WWWPages/EpicsDoc.html

\bibitem{ehs} JLAB EH\&S Manual. URL: http://www.jlab.org/ehs/ehsmanual/.

\bibitem{pss} JLAB/CEBAF Personal Safety System (PSS) Manual. URL: http://www.jlab.org/accel/ssg/user\_info.html

\bibitem{ops} Accelerator Operations Directive. URL: http://www.jlab.org/ops\_docs/online\_document\_files/ACC\_online\_files/\\accel\_ops\_directive.pdf (URL is available inside JLAB site). 

\end{thebibliography}

\end{document}

